\begin{abstract}
Tracking fast-moving objects in unconstrained natural videos remains a challenging problem due to background dynamics, illumination changes, camera motion, and scale variation. Classical motion-based pipelines offer efficiency and interpretability, whereas modern deep-learning-based trackers achieve high performance at greater computational cost. This work revisits classical computer vision approaches for multi-object tracking in the context of bird detection and tracking by systematically designing and evaluating an OpenCV-based pipeline against a state-of-the-art deep tracker, NetTrack, on the BFT dataset. The proposed pipeline incorporates camera motion compensation, adaptive motion-based detection, region refinement, and Kalman-filter-based data association. Quantitative experiments using IoU, HOTA, MOTA, and IDF1 demonstrate that the classical pipeline achieves competitive results in simpler scenarios, while deep-learning-based methods consistently outperform it in highly dynamic and cluttered environments. These findings reveal a clear trade-off between tracking accuracy and computational efficiency.
\end{abstract}
